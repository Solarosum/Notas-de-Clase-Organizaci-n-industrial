
%Tipo de documento
\documentclass[letterpaper,12pt,twocolumn]{report}

%Margenes
\usepackage[top=2.5cm, bottom=2.5cm, left=2.5cm, right=2.5cm]{geometry}

%Paquetes variados
\usepackage[T1]{fontenc}
\usepackage[utf8]{inputenc}
\usepackage[spanish]{babel}
\usepackage{lmodern}
\usepackage{amsmath}
\usepackage{amsfonts}
\usepackage{amssymb}
\usepackage{amsthm}
\usepackage{graphicx}
\usepackage{color}
\usepackage{xcolor}
\usepackage{url}
\usepackage{theorem}
\usepackage{textcomp}
\usepackage{hyperref}
\usepackage{parskip}
\usepackage{fancyvrb}
\usepackage{tikz}
\usepackage[most]{tcolorbox}
\usepackage{tabularray}
\usepackage{mathtools} 

%Diagramas de arbol
\usepackage{forest}

%Capitulos elegantes
\usepackage[Rejne]{fncychap}

%Datos 
\title{Notas de Clase: Organización industrial}
\author{Fabián Méndez Martinez}
\date{Agosto 2024 - Enero 2025}
%Comandos
\renewcommand{\baselinestretch}{1.3}

%Documento
\begin{document}

%Presentación
\maketitle
\tableofcontents

%Contenido

\chapter{Parte Uno}


%Datos de la materia 
\begin{tcolorbox}[title=\large Información de la EE]
    \begin{itemize}
        \item Nombre: Organización industrial
        \item NRC: 92681
        \item Clave: ECTE - 38008
        \item Créditos: 8
    \end{itemize}
\end{tcolorbox}

%Profesor
\begin{tcolorbox}[title=\large Información del Profesor]
    \begin{itemize}
        \item Profesor: Miguel Rodrigo Ávila
Pulido
        \item Correo: miavila@uv.mx
    \end{itemize}
\end{tcolorbox}

%horario
\begin{table}[h]
\centering
\begin{tabular}{l|ll}
\multicolumn{3}{c}{\textbf{Horario}}      \\ 
\hline
Martes     & 7:00 & 9:00          \\ 
\hline
Jueves & 7:00 & 9:00          \\ 
\hline
Viernes   & 9:00 & 11:00          \\ 
\hline
\multicolumn{3}{c}{Aula: 217}   
\end{tabular}
\end{table}
\pagebreak

%Clases
\section{Clase del 19 de agosto}

\subsection*{Organización industrial}

La organización industrial estudia la competencia imperfecta. En las economías de mercado se tiene que 

\begin{itemize}
    \item Las firmas deciden qué y cuanto producir
    \item Los consumidores eligen comprar en la mejor alternativa por precio, calidad, ubicación, etc.
\end{itemize}

Este curso estudia el comportamiento de las empresas.

\subsection*{Historia de la Organización industrial}

\subsubsection*{Harvard}

\textbf{Caso: Demanda del gobierno de EE.UU contra \textit{US Steel}:} La firma concentraba el $70\%$ del mercado de producción de acero. El gobierno americano perdió la demanda, \textit{US Steel} no violó la ley de competencia pues \textit{"La Ley no hace al tamaño una ofensa"}.

¿Comó inferis comportamientos ilegales a partir de características estructurales como el tamaño? A partir de esto se creo el siguiente marco

\begin{tcolorbox}[title=Definición]
$\text{Marco}=\text{Estructura}\rightarrow \text{Conducta}\rightarrow \text{desempeño}$
\end{tcolorbox}


La estructura determina la forma en la que las firmas interactuan entre sí, con los compradores y los potenciales entrantes.

\begin{tcolorbox}[title=Definición]
Estructura $ = f$(Firmas, tecnología, productos, etc.)
\end{tcolorbox}

La estructura determina la conducta (forma en la que las firmas se comportan en una estructura de mercado dada). La conducta determina el desempeño (Beneficios, excedentes, bienestar)

Señalan que la alta concentración de mercado reduce el excedente de los consumidores.

\textbf{Debilidades:} Suponer que la concentración es exógena, no tomar en cuenta diferencias interindustriales.

\subsubsection*{Chicago}

Está de acuerdo con \textbf{Harvard} en que las firmas con mayor concentración de mercado tienden a generar mayores beneficios. Para \textbf{Harvard} esto implica que los mercados concentrados son menos competitivos.

Para \textbf{Chicago} este no es el caso, puede ser que las firmas con mayor de mercado sean más eficientes y por ello tengan mayores beneficios.

\textbf{Énfasis en teoría de precios:} Afirman que el comportamiento monopólico es dificil de confirmar, no ocurre con frecuencia y cuando ocurre es tipicamente transitorio.

\subsubsection*{Post Chicago}

Énfasis en la toma de decisiones estratégicas, el comportamiento de las firmas se representa con modelos matemáticos y de toería de Juegos. Se construye un marco formal que supera el debate entre \textbf{Chicago} y \textbf{Harvard}.

También surge nueva literatura empírica que combina los modelos matemáticos con pruebas econométricas (\textit{Ordered-probit, multinomial-logit, random coefficient logit, etc.}) computacionalmente demandantes.

\textbf{Dificultades:} ¿Qué modelos usar para cada caso? Efectos de segundo orden.

\subsection*{Competencia Perfecta}

\begin{tcolorbox}[title= Supuestos]

Un agente se comporta de forma competitiva si:

\begin{itemize}
    \item Suponen que el precio de mercado está dado.
    \item Bienes homogeneos.
    \item Libre entrada y salida, por ejemplo los costos fijos pueden ser considerados una barrera de entrada.
    \item Información perfecta.
    \item No hay externalidades.
    \item Bienes perfectamente divisibles.
    \item No hay costos de transacción.
\end{itemize}

\end{tcolorbox}


\section{Clase del 21 de agosto}

\subsection*{El Benchmark de competencia perfecta}

\textbf{Objetivo de la firma:} Maximizar beneficios por lo que se tiene que 

\begin{tcolorbox}[title=Precio en competencia perfecta]
    $$\text{max}=pq-C(q)$$
Obteniendo F.O.C
    $$ p-C'(q)=0 $$
    $$ p=C'(q)$$
    $$ p=MC(q) $$
\end{tcolorbox}

Se tienen los siguientes funciones de costo

\begin{itemize}
    \item Costo total: $C(q)=c(q)+F$
    \item Costo variable: $c(q)$
    \item Costo medio: $AC=\frac{c(q)}{q}+\frac{F}{q}$
    \item Costo marginal: $MC= C'(q)$
    \item Costo variable medio: $AVC=\frac{c(q)}{q}$
    \item Costo fijo medio: $AFC=\frac{F}{q}$
\end{itemize}



% Aquí va la gráfica 1.....


\textbf{Demuestre} que la curva de costo marginal cruza la curva de costo medio en su mínimo.

Sea $\hat{Q}$ la cantidad que minimiza $AC:\frac{dAC(\hat{Q})}{d\hat{Q}}=0$ se tiene que 

$$
\dfrac{AC(Q)}{dQ}= \frac{dC(Q)}{dQ}\cdot Q^{-1}+ C(Q)(-1)Q^{-2},
$$

$$ 
\vdots
$$

$$
\dfrac{AC(Q)}{dQ} = \dfrac{\frac{dC(Q)}{dQ}Q-C(Q)}{Q^{2}}.
$$

Si se tiene que $Q=\hat{Q}$, se cumple que 

$$ \frac{dAC(\hat{Q})}{d\hat{Q}}=C(\hat{Q}),$$

por lo que 

$$ \dfrac{AC(Q)}{dQ} = \dfrac{C(\hat{Q})-C(\hat{Q})}{\hat{Q}^{2}} = 0.$$

\subsection*{Mercado competitivo}

\subsubsection*{Equilibrio de corto plazo}

Pensemos en $n$ firmas idénticas y que los costos fijos son costos fijos son costos hundidos en el corto plazo. La oferta de mercado de corto plazo es la suma horizontal sw curvas de oferta de cada firma

% Aquí van las gráficas 2 y 3 side by side

En el corto plazo puede ser el caso que una firma representativa tenga beneficios mayores a cero.

\subsubsection*{Equilibrio de largo plazo}

La entrada de firmas al mercado en el largo plazo se detiene cuando el precio llega al mínimo del costo promedio se largo plazo.

% Aquí va las gráficas 4 y 5

\section{Clase del 22 de agosto}


\subsection*{Recordatorio de bienestar y excedentes}

Se define al \textbf{Bienestar} como la suma de los excedentes de todos los consumidores y de todos los productores. El \textbf{excedente del consumidor} es el área bajo la curva inversa de demanda y por arriba del precio (i.e diferencia entre lo que un consumidor esta dispuesto a pagar y el precio que efectivamente pagó multiplicado por la cantidad consumida).


El \textbf{Excedente del productor} es la área de la curva de oferta y abajo del precio.


% Aquí va la gráfica 5

\subsection*{Teoremas del bienestar}

\begin{tcolorbox}[title= Primer teorema del bienestar]
   
   Es cualquier asignación descentralizada obtenida a tráves de un mercado libre es eficiente en el sentido de pareto y maximiza el bienestar social.
   
\end{tcolorbox}

\begin{tcolorbox}[title= Segundo teorema del bienestar]
   
   Es cualquier asignación en la curva de contratos es alcanzable a través de la redistribución de las asignaciones iniciales.
   
\end{tcolorbox}

\subsection*{Teoría de Juegos }

Estudia los problemas de decisión multi-agente. En competencia imperfecta hay interacciones estratégicas entre agentes. 

\subsubsection*{Juego en forma normal}

\begin{enumerate}
    \item Conjunto de Jugadores $I=\{1,2,\dots,N\}$
    \item Para cada jugador un conjunto de estrategias $A_i$.
\end{enumerate}

De esta ultima definimos que sea $a=(a_1,a_2,\dots,a_n)$ una lista de las acciones de cada jugador. Decimos que $a$ es un perfil de acciones.

\begin{enumerate}
	\setcounter{enumi}{2}
    \item Para cada jugador una función de pago.
\end{enumerate}


\section{Clase del 26 de Agosto}

\subsection*{Ejemplo de juego en forma normal}

Suponga que en competencia en precios cuando los bienes son sustitutos perfectos

\begin{enumerate}
	\item Conjunto de Jugadores: $I: \{f1, f2\}$
	\item Conjunto de estrategias: $ P_i \in R_{++}$
	\item Conjunto de funciones de pagos: $\Pi_i(P_i,P_j) = (P_i-C_i)q_i(P_i,P_j)$
\end{enumerate}

\subsection*{Dilema del prisionero}

\begin{enumerate}
	\item Conjunto de Jugadores: $I: \{\text{Prisionero }1,\text{Prisionero } 2\}$
	\item Conjunto de estrategias: $ S_i= \{ \text{Delatar, no delatar} \}$ para $i \in I$
	\item Conjunto de funciones de pagos: 
	
	$ U_i(S_i,S_2) \begin{cases}
		-5 & \text{ si } S_i=M,S_j=F \\
		 -4 & \text{ si } S_i=S_j=F\\
		  -2 & \text{ si } S_i=S_j=M\\
		   -1 & \text{ si } S_i=F,S_j=M
	\end{cases}
	$
\end{enumerate}

\subsection*{Representación matricial de Juegos Finitos con dos jugadores}

\textbf{Convenciones:} Las filas representan al jugador 1, las columnas al jugador 2 y en las entradas los pagos.

\subsubsection*{Matriz del dilema del prisionero}
$$
\begin{matrix}
	 &  F & M \\
	F & (-4,-4) & (-1,-5)  \\
	M & (-5, -1) & (-2,-2)
	 
\end{matrix}
$$

\subsection*{Equilibrio en estrategias dominantes}

Sea $s\in \Pi^{N}_{j=1}S_j$ para $i\in\{1,\dots, N \}$ denotemos a $S_{-i} = (S_i,S_2,\dots,S_{i-1},S_{i+1},\dots ,S_n)$ y reescribimos el conjunto de acciones $S$ como $S= (S_i,S_{-i} )$.

En los juegos en forma normal una estrategia es una acción que pertenece al conjunto de acciones de algún jugador.

\begin{tcolorbox}[title= Definición]
	Si para todo $s_{-i} \in \Pi_{j\neq i}S_{-i}$ y para todo $s_i \in S_i$ sujeto a que $ s_i \neq \widetilde{s_i} $
	
	$$ \Pi_i(\widetilde{s_i},s_{-i})> \Pi_i(s_i,s_{-i}) $$

Se tiene que $\widetilde{s_i} \in S_i$ es una estrategia estrictamente dominante para el jugador $i$.
	
\end{tcolorbox}

 
En el dilema del prisionero la estrategia estrictamente dominante de ambos jugadores es delatar (F).

\subsection*{Una primera definición de equilibrio}

\begin{tcolorbox}[title= Primera definición de equilibrio]
	Un perfil de estrategias $s=(\widetilde{s_1},\widetilde{s_2}, \dots \widetilde{s_n}) \in \Pi_{j=1}^N S_j$ es un equilibrio en estrategias dominantes si $\widetilde{S_i}$ es una estrategia dominante para el jugador $i$.
\end{tcolorbox}

Para el dilema del prisionero (F,F), $U_i=-4$ para $i\in I$. Por lo que no hay razón para que las desiciones de mercado lleven a un óptimo de pareto.

\subsubsection*{La batalla de los sexos}


$$
\begin{matrix}
	&  Opera & Football \\
	Opera & (2,1) & (0,0)  \\
	Football & (0, 0) & (1,2)
	
\end{matrix}
$$

No hay equilibrio en estrategias estrictamente dominantes.

\begin{tcolorbox}[title= Segunda definición de equilibrio]
	\textbf{Equilibrio de Nash}: Un perfil de estrategias $\widetilde{S}=(\widetilde{S_1},\widetilde{S_2}, \dots, \widetilde{S_n}) \in \Pi_{j=1}^N S_j$ es un equilibrio de Nash si ningún jugador tiene incentivos a desviarse de este perfil estratégico. Esto es: si para todo $i \in I$ para todo $s_i \in S_i$, 
	
	$$ \Pi(\widetilde{S_i},\widetilde{S_{-i}})\geq \Pi_i(S_i,\widetilde{S_{-i}}) $$
\end{tcolorbox}

\section{Clase del 28 de agosto}

\subsection*{¿Cómo encontrar el equilibrio de Nash?}

Tengamos un juego simple $2x2$ con una matriz 

$$
\begin{matrix}
	&  C & D \\
	C & (-1,-1) & (-10,0)  \\
	D & (0, -10) & (-6,-6)
	
\end{matrix}
$$

Por descripción tengamos que 

\begin{itemize}
	\item $(C,C)$: No es un equilibrio de Nash, ambos jugadores no tienen incentivos para jugar \textit{D}.
	
	\item $(C,D)$: No es equilibrio de Nash, el jugador $1$ tiene incentivos para jugar \textit{D}.
	
	\item $(D,C)$: No es equilibrio de Nash, el jugador $2$ tiene incentivos para jugar \textit{D}.
	
	\item $(D,D)$: Sí es equilibrio de Nash.
\end{itemize}


\subsubsection*{Ejemplo}

$$
\begin{matrix}
	&  C & D \\
	C & (2,1) & (0,0)  \\
	D & (0,0) & (1,2) 
	
\end{matrix}
$$

\begin{itemize}
	\item $(O,O)$: Sí es equilibrio de Nash

	\item $(F,F)$: Sí es equilibrio de Nash
\end{itemize}


Si el juego más complejo, resolverlo por inspección tomaría mucho tiempo.

\subsubsection*{Análisis de Mejor Respuesta}

\begin{itemize}
	\item Sea $i \in \{1,2,\dots , N \}$ y $S_{-i} \in \Pi_{j\neq1} S_j$
	\item Sea $BR_i(S_{-i})=\text{ arg.max } U_i(s_i,S_{-i})$
	\item Si $S_{-i} \in \Pi_{j\neq1} S_j \rightarrow \text{ Si }BR_i(S_i)$ es la correspondencia de mejor respuesta del jugador $i$.
\end{itemize}

\begin{tcolorbox}[title= Nota]
	
Varios valores de $s_i$ podrían resolver el problema de maximización. \textit{arg.max} es el conjunto de valores de "$x$" para los que la función alcanza su valor máximo.

Un perfil estratégico $$\widetilde{S}=(\widetilde{s_1},\widetilde{s_2},\dots, \widetilde{s_N})\in \Pi_{j=1}^N S_j,$$ es un equilibrio de Nash si para todo i: 

$$ \widetilde{S_i}\in BR_i(S_{-i}). $$

\end{tcolorbox}


Cada jugador elige una estrategia que es su mejor respuesta a las estrategias de equilibrio de sus rivales.

Para resolver un juego encontrando el equilibrio de Nash con análisis de mejor respuesta:

\begin{enumerate}
	\item Para cada jugador $i$, buscar la correspondencia de mejor respuesta $BR_i(\cdot)$
	\item Buscar las intersecciones de las mejores respuestas, con esto obtendrás todos los equilibrios de Nash.
\end{enumerate}

\subsubsection*{Ejemplo}
Considere la competencia en precios con productos diferenciados $c_1=c_2=0$.
Sea la función de pagos $\Pi_i(P_i,P_j)=(P_i-C_i)q_i(P_i,P_j),$ suponga $q_i=1-2P_i+P_j$. Encuentre el equilibrio de Nash:

%No sé si el uso de align sea la mejor solución

Partiendo desde la función de pagos
\begin{align*}
	\Pi_i(P_i,P_j) &=(P_i-C_i)\cdot q_i(P_i,P_j),\\
	\shortintertext{sustituyendo $q_i$}
	\Pi_i(P_i,P_j) &=(P_i-C_i) \left( 1-2P_i+P_j\right),\\
		\shortintertext{considerando $C_i=0$}
	\Pi_i(P_i,P_j) &= P_i \left( 1-2P_i+P_j\right) ,\\
	\Pi_i(P_i,P_j) &= P_i-2P_i^2+P_iP_j.
\end{align*}
\vspace{.75cm}
Calculando las F.O.C
\begin{equation*}
	\dfrac{\partial \Pi_i(P_i,P_j)}{\partial q_i} = 
	1-4P_iP_j=0,
\end{equation*}

se sigue que

\begin{align*}
1-4P_iP_j&=0, \\
4P_i+P_j&=1, \\
4P_i&=1-P_j,\\
P_i&=\frac{1-P_j}{4}.
\end{align*}
Por lo que la función de mejor respuesta esta dada por 

$$ BR_i= \frac{1-P_j}{4}.$$


El proceso es análogo para el caso de $P_j$, sustituyendo $P_j$ en $P_i$

\begin{align*}
	P_i&=\frac{1-\left( \frac{1-P_i}{4} \right) }{4},\\
	P_i&=\frac{\frac{4-1+P_i}{4} }{4},\\
	P_i&=\frac{3+P_i}{16},
\end{align*}

% Separado para que haya una mejor distribución

\begin{align*}
	P_i&=\frac{3+P_i}{16},\\
	16P_i&=3+P_i,\\
	15P_i&=3,\\
	P_i &= \dfrac{3}{15}=\dfrac{1}{5}.
\end{align*}

Análogamente se se observa que 

$$ P_i = P_j = \dfrac{1}{5}. $$

\section{Clase del 29 de agosto}

\subsection*{Comentarios}

\begin{enumerate}
	\item Un equilibrio en estrategias dominantes es siempre un equilibrio de Nash.
	\begin{itemize}
		\item Una estrategia dominante es la mejor respuesta a cualquier perfil estratégico.
		\item Un equilibrio de Nash no siempre es equilibrio en estrategias dominantes.
	\end{itemize}
	\item Al buscar el equilibrio de Nash, solo considera desviaciones unilaterales del perfil estratégico.
\end{enumerate}


\begin{tcolorbox}[title=Fundamentos del equilibrio de Nash]
	\begin{itemize}
		\item Todos los agentes son racionales y todos los agentes saben que son racionales.
		\item Como los agentes son racionales, harán lo posible por maximizar su utilidad dadas las acciones de sus rivales.
		\item Al final se jugará un perfil estratégico del por que ningún agente se desviará.
	\end{itemize}
\end{tcolorbox}

\subsubsection*{¿Cómo elegir entre varios equilibrios de Nash?}

\begin{itemize}
	\item Ocasionalmente uno es Pareto dominante
	\item Otras veces uno actúa como punto focal
\end{itemize}

En juegos repetidos, el proceso de aprendizaje lleva a un equilibrio de Nash, incluso sin agentes racionales. La historia, dependiendo del punto de partida, podrías converger a equilibrios de Nash distintos.
$$
\begin{matrix}
  &	Piedra & Papel & Tijera \\
  Piedra & (0,0) & (-1,1) & (1,-1) \\
  Papel & (1,-1) & (0,0) & (-1,1) \\
  Tijera & (-1,1) & (1,-1) & (0,0)
\end{matrix}
$$

\textbf{No hay equilibrio de Nash en estrategias puras}
$$
\begin{matrix}
	&	Divide & Roba \\
	Divide & (\frac{1}{2},\frac{1}{2}) & (0,1) \\
	Roba & (1,0) & (0,0)
\end{matrix}
$$

En \textit{(Divide, Roba), (Roba, Divide), (Roba, Roba)}.

\subsection*{Juegos de forma extensiva}
\subsubsection*{Juego del piloto vs terrorista}

% Poner esa coma fue un martirio

\begin{forest}
	[El piloto mueve [Volar a cuba [Bomba [$(1\text{,}-1)$] ] [No bomba[$(1\text{,}1)$]]][Volar a NYC[Bomba[($-1\text{,}-1)$] ] [No Bomba [$(2\text{,}0)$]]]]
\end{forest}


$$
\begin{matrix}
	& BnNB & NB,B & B,B & NB,NB \\
	Cuba & (-1,-1) & (1,1) & (-1,-1) & (1,1)\\
	NYL & (2,0) & (-1,-1) & (-1,-1) & (2,0)
\end{matrix}
$$

\subsection*{Equilibrios de Nash}

\begin{itemize}
	
	\item \textit{(Cuba,(NB,B))}
	\item \textit{(NYL,(B,NB))}
	\item \textit{NYL(NB,NB)}
\end{itemize}


\section{Clase del 2 de septiembre}

Consideremos el primer equilibrio, un piloto racional diría, si vuelo a NYC, es interés del terrorista cambiar a \textit{NB} en lugar de \textit{B}. Este razonamiento no es capturado por el equilibrio de Nash:

\begin{center}
\noindent \textbf{Algunos equilibrios de Nash se sostienen por amenazas no creíbles.}
\end{center}

Necesitamos un concepto de equilibrio que elimine amenazas no creíbles.

\subsection*{Inducción hacia atrás (Backward induction)}

\begin{enumerate}
	\item Empieza resolviendo para las desiciones óptimas en el nodo terminal, encuentre los pagos.
	\item Vaya un paso atrás, resuelva para las desiciones óptimas, anticipando que todos se comportan racionalmente en nodos subsecuentes, encuentre los pagos.
	\item Itere hasta alcanzar el modo inicial.
\end{enumerate}

% Aquí va el la solución del arbol


\textbf{Equilibrio:} \textit{(NYC, (NB,NB))}.

\subsection*{Equilibrio Perfecto en subjuegos}

\begin{tcolorbox}[title=Definición]
	Un subjuego es un nodo de decisión del juego original con los nodos de decisión y los nodos terminales que siguen directamente a este nodo.
\end{tcolorbox}

Un subjuego es llamado subjuego estricto si es distinto del juego original.

\begin{tcolorbox}[title=Definición]

Un perfil estratégico es un equilibrio perfecto en subjuegos si induce un equilibrio Nash en cada subjuego del juego original.

\end{tcolorbox}

\subsubsection*{Suponga un juego finito:}
Incluso si los jugadores mueven simultáneamente el juego puede resolverse por \textit{backward induction}.

\begin{enumerate}
	\item Inicie por los subjuegos más profundos y encuentre el equilibrio de Nash en ellos.
	\item En la forma extensiva, remplaza los subjuegos más pequeños por los pagos del equilibrio de Nash.
	\item Itera hasta que no queden subjuegos.
\end{enumerate}

 \subsection*{Resumen}
 
 \begin{tcolorbox}[title= Juego en forma normal]
 	Tres conjuntos \textit{(jugadores, acciones, pagos)}
 	
 	\begin{itemize}
 		\item Solución: Equilibrio perfecto en subjuegos / equilibrio de Nash
 		\item Prueba y error / función de mejor respuesta
 	\end{itemize}
 	
 \end{tcolorbox}
 
 \begin{tcolorbox}[title= Juegos en forma extensiva]
 	\begin{itemize}
 		\item Concepto de solución: equilibrio perfecto en subjuegos (elimina amenazas no creíbles)
 		\item Método: Con periodos finitos usar inducción hacia atrás
 	\end{itemize}
 \end{tcolorbox}
 
 
\chapter{Parte II}

\section{Clase del 9 de Septiembre}

\section*{Monopolio de bienes durables}

Bienes que solo se compran cada $\pm3$ años y se usan todo ese periodo (celulares, pantallas, refrigeradores, autos).

\subsection*{Conjetura de Coase}

Suponga que un monopolista tiene toda la tierra del mundo y quiere venderla al mayor beneficio descontado posible.

\begin{itemize}
	\item Año 1: El monopolista vende la mitad de la tierra a $P^M_1$ (i.e demanda lineal con MC=0)
	\item Año 2: El monopolista quiere vender lo mismo pero, a menos que la población crezcaa muy rápido, la demanda será menor $P^M_2<P^M_1$
\end{itemize}

\begin{tcolorbox}
	"Si los consumidores no descuentan mucho el tiempo y esperan que el precio baje en el futuro, el monopolio de bienes durables cobra un precio menor que el monopolio tradicional"
\end{tcolorbox}

\textbf{Modelo A:} Continuo de consumidores, demanda con pendiente negativa, consumidores viven dos periodos, costo marginal cero. Demanda agregada de un periodo por los servicios del bien $P=100-q$

\textbf{Juego:}

\begin{itemize}
	\item Jugadores: Consumidores y monopolista.
	\item Conjunto de acciones: Vendedor elige $q_1$ y $q_2(q_1)$, compradores eligen si comprar en cada periodo.
	\item Pagos: $PS$, $CS$ (no se descuentan).
\end{itemize}

\textbf{Buscamos el equilibrio perfecto en subjuegos:} La técnica a utilizar es backward induction. 

\textbf{En el periodo 2:} Suponga que se vendió $\bar{q_1}$ en el primer periodo los consumidores que consumieron en el primer periodo lo vuelvan a comprar.

$$\text{Demanda periodo 2} \rightarrow P=100-\bar{q_1}-q_2$$

Recordemos la condición de maximización del monopolio ($MR=MC$)

\begin{align*}
	B =& (100-\bar{q_1}-q_2)(q_2) \\
	B =& 100q_2-\bar{q_1}q_2-q_2^2 \\
	MR =& 100-\bar{q_1}-2q_2 = 0 \\
	MR =& 100-\bar{q_1}=2q_2\\
	MR = & \dfrac{100-\bar{q_1}}{2}=q_2
\end{align*}

Sustituyendo en la función de la inversa de demanda

\begin{align*}
	P_2 = & 100-\bar{q_1} - \dfrac{100-\bar{q_1}}{2}\\
	P_2 =& 100-\bar{q_1}-50+\dfrac{\bar{q_1}}{2}\\
	P_2=& 50-\dfrac{\bar{q_1}}{2}
\end{align*}

Obteniendo los beneficios

\begin{align*}
	\Pi_2 =& \left( 50-\dfrac{\bar{q_1}}{2}\right) \left( 50-\dfrac{\bar{q_1}}{2}\right)  \\
	\Pi_2 =& \left( 50-\dfrac{\bar{q_1}}{2}\right) ^2
\end{align*}

Suponga que el monopolista vende $\bar{q_1}$ en $t=1$ ¿Cuál es el máximo $P$ que puede cobrar?

El consumidor marginal debe estar indiferente entre comprar en $t=1$ y $t=2$
Excedente del consumidor que compra en el primer periodo

$$
2(100-\bar{q_1})-P_1
$$

Excedente del consumidor que compra que compra en el segundo periodo 

$$ 100-q_1-P_2$$

Igualando se tiene que 

\begin{align*}
	50- \dfrac{\bar{q_1}}{2} =& 200- 2\bar{q_1}-P_1\\
	2\bar{q_1}-\dfrac{\bar{q_1}}{2}=&  150-P_1\\
	\dfrac{4\bar{q_1}-\bar{q_1}}{2}=&  150-P_1	\\
	\dfrac{3\bar{q_1}}{2}=& 150-P_1\\
	150-\dfrac{3\bar{q_1}}{2} =& P_1
\end{align*}

Calculando los beneficios de ambos periodos


$$Max_{q_1}(\Pi_1+Pi_2)= \left( 150-\dfrac{3\bar{q_1}}{2}\right)q_1 + \left( 50-\dfrac{\bar{q_1}}{2}\right) ^2
$$


\section{Clase del 11 de septiembre}

\textbf{Estrategia alternativa:} Rentar los bienes. Vender equivale a cobrar un único precio por periodo de tiempo indefinido. Rentar es cargar un precio por usar un bien un definido.

\begin{itemize}
	\item Demanda agregada: $ P=100-q $
\end{itemize} 

\textbf{Resolveremos el modelo estático dos veces}

\textbf{Para el periodo 1}

$$ P=100-q $$

Calculando beneficios

$$R = (100-q)q,$$

$$R = 100q-q^2$$

Derivando e igualando a cero

\begin{align*}
	\frac{\partial R}{\partial q} &= 100-2q\\
	0 &= 100-2q, \\
	2q &= 100, \\
	\frac{100}{2} &= q\\
	50 &= q.
\end{align*}

Sustituyendo en la inversa de demanda

$$ P=100-q, $$
$$ P= 100-50, $$
$$ P=50.$$

\textbf{¿La firma prefiere vender o rentar bienes? La firma prefiere rentar.}

\subsection*{Oligopolio estático: Competencia de Bertrand}

Método para estudiar oligopolio que bajo ciertas condiciones resulta en un equilibrio de Nash en un juego de precios.

\begin{itemize}
	\item $2$ Firmas
	\item $q(P)$ demanda total por el bien homogéneo
	\item $ q(\cdot) < 0$
	\item  Consumidores consumen del vendedor más barato.
	\item Si los vendedores cobran lo mismo, los consumidores se dividen $50/50$
\end{itemize}


Sea la demanda de la firma $q_i$

$$ q_i(P_1,P_2) \begin{cases}
	
	q(P_1) \text{ si } P_1<P_2 	\vspace{.25cm}  \\
	
\vspace{.25cm}	\dfrac{q(P)}{2} \text{ si } P_1=P_2 \\

	0 \text{ si } P_1>P_2	
\end{cases}$$

\textbf{Juego en forma normal:}

\begin{itemize}
	\item Jugadores: $I= \{ \text{Firma 1, Firma 2} \}$
	\item Estrategias: $s_i=\{ P_1,P_2 \} | P_1,P_2\in R_{+}$
	\item Pagos: $\Pi_i=(P_i-C_i)\cdot q_i(P_1,P_2)$
\end{itemize}

\textbf{Concepto de equilibrio: Equilibrio de Nash}

Un par $ \{ P_1^B, P_2^B \} $ es un equilibrio de Bertrand (Nash en precios) si:

\begin{enumerate}
	\item  Dado $P_2$
\end{enumerate}


\section{Clase del 12 de septiembre}

\subsection*{Bertrand con costos por cambiar de proveedor}

En algunas industrias los consumidores deben pagar por cambiar de proveedor. 

\begin{itemize}
	\item Compatibilidad (\textit{Google}, \textit{Microsoft})
	\item Costos de transacción (Acceso a internet)
	\item  Costos de aprendizaje
	\item Costos psicológicos (lealtad de marca)
\end{itemize}

\begin{tcolorbox}[title=Supuestos]
	\begin{itemize}
		\item 2 firmas
		\item 2 periodos
		\item Bien homogéneo
		\item Costos constantes $c$ por periodo
		\item Consumidores forward-looking
		\item Masa de consumidores 1 
		\item Utilidad bruta $U$ por consumir el bien por cada periodo
		\item No hay tasa de descuento
	\end{itemize}
\end{tcolorbox}

En el periodo $2$ si un consumidor compró a la firma $i$ en $t=1$ debe pagar $\sigma$ para comprar a $j$ en $t=2$. Suponga que $\sigma>U-c$ (costo por cambiar es suficientemente grande)

\subsubsection*{Tiempo}

\begin{itemize}
	\item $t=1$: Firmas $1$y $2$ eligen $P_1^1$ y $P_2^1$ simultáneamente.
	\item  $t=1.5$: Consumidores eligen sus proveedores (avientan una moneda)
	\item $t=2$ Firmas $1$ y $2$ eligen precios $P_1^2$ y $P_2^2$ simultáneamente.
	\item $t=2.5$: Consumidores eligen proveedores
\end{itemize}

\subsubsection*{Resolución}

Iniciamos en el $t=2$. Sea $s_i$ el market share de la firma $i$ en  $t=1$.

$$ P_i^2 \leq P_j^2-\sigma, $$

como la firma $j$ quiere vender una cantidad positiva, su precio debe ser 

$$P_j^2\leq U,$$

debe ser el caso que $P_i$:

$$P_j^2\leq U-\sigma \leq c, $$

si los costos por cambiar de producir son altos, no es rentable para la firma $i$ ganar market share en $t=2$. Ambas firmas se concentran en sus consumidores y cobran $P_i² = U.$ Los beneficios son

$$\Pi_i=(U-c)s_i$$

Para $t=1.5$ Los consumidores anticipan que tendrán excedente cero en el segundo periodo. Se sigue que los consumidores compran del proveedor más barato en $t=1.5$

$$s_i(P_i^1,P_j^1)= \begin{cases}
	1 \text{ si } P_i^1<P_j^1\\
	\dfrac{1}{2} \text{ si } P_i^1=P_j^1\\
	0 \text{ si } P_j^1<P_i^1
\end{cases} $$

con beneficios

$$ \Pi_i^1 (P_1^1,P_2^1)= \begin{cases}
	P_1-c \text{ si } P_i^1<P_j^1\\
	\dfrac{P_i-c}{2} \text{ si } P_i^1=P_j^1\\
	0 \text{ si } P_j^1<P_i^1
\end{cases} $$

Para obtener los precios se tiene que 

$$\Pi_i(P_i^1,P_j^1)= \begin{cases}
	P_i^1-2c+U | P_i^1<P_j^1, P_i^1\leq U \\
	\dfrac{P_i^1+U}{2}-c \text{ si } P_i^1=P_j^1\\
	0 \text{ si } P_j^1<P_i^1
\end{cases} $$

Por conocimientos de Bertrand o por casos se llega que el precio es

$$P_i^1=2c-U,$$

esto es el Costo marginal percibido por los consumidores.

\subsection*{Competencia de Cournot}

Análisis de oligopolio que bajo ciertas condiciones es un equilibrio de Nash en un juego de cantidades

\begin{tcolorbox}[title= Supuestos]
	\begin{itemize}
		\item Firmas eligen nivel de producción
		\item $2$ Firmas
		\item Costos: $C_i(q_i)=c_iq_i$ $|$ $ i=1,2$
		\item Demanda: $P(Q)= a-bQ$ $|$ $Q=q_1+q_2$
	\end{itemize}
\end{tcolorbox}

\subsubsection*{Juego en forma normal}

\begin{itemize}
	\item Jugadores: $N= \{\text{Firma 1, Firma 2}\}$
	\item Estrategias: $S_i=q_i \in R_+$
	\item Pagos: $\Pi_i(q_1,q_2)= P(Q)q_i-c_i(q_i)$ para $i \in N$
\end{itemize}

Un par $(q_i^c,q_2^2)$ es un equilibrio de Cournot si:

Dado $q_2= q_2^c, q_1^c$ resuelve $\text{max}_{q_{1}} \Pi_1(q_1,q_2^c)$.

Dado $q_1=q_1^c,q_2^c$ resuelve $\text{max}_{q_{2}}\Pi_2(q_1^c,q_2)$

Dado que los rivales juegan la estrategia de equilibrio de Cournot en cantidades, ninguna firma puede elevarse beneficios cambiando su nivel de producción.

Los correspondientes niveles de precio en el equilibrio de Cournot son:

$$ p^c=a-b(q_1^c+q_2^c) $$

\subsubsection*{Extensión al caso con n firmas}

Cada firma maximiza beneficios de acuerdo a: 

$$ \Pi(q_1,q_2,\dots,q_n)= P(Q)q_i-c_i(q_i) $$

La función de mejor respuesta para $i$ es 

\section{Clase del 19 de septiembre}

\subsection*{Análisis de Bienestar}


El excedente del consumidor es el área entre la curva de demanda y el preciode mercado.

Calcule el excedente del consumidor para el ejemplo anterior sería 

$$ CS= \frac{[N(a-c)]^2}{2b(N+1)^2} $$

para el caso del productor se tiene que 

$$  PS= \frac{N(a-c)^2}{b(N+1)^2}  $$

Siendo el bienestar 

$$ W= \frac{N(a+c)^2(N+2)}{2b(N+1)^2} $$

\subsection*{Actividades opcionales}

\begin{itemize}
	\item ¿Qué ocurre con el bienestar cuando el número de firmas aumenta?
	\item ¿Qué ocurre en Cournot con costos asimétricos?
	\item En Cournot con costos asimétricos, obtenga una expresión para el indice de lerner que relacione el Market share de la firma $i$ con la elasticidad inversa.
	
\end{itemize}

	\subsection*{Cournot con Movimientos Secuenciales (Stackelberg)}

Misma estructura que el modeloen un solo tiempo. Las firmas mueven de forma secuencial.

\begin{tcolorbox}[title= Tiempos]
	\begin{itemize}
		\item $t=1$ Firma $1$ elige $q_1$ (líder)
		\item $t=2$ Firma $2$ elige $q_2$ (seguidora)
	\end{itemize}
\end{tcolorbox}

Resolveremos con Backward induction

\begin{tcolorbox}[title=Información necesaria]
	\begin{itemize}
		\item $ P(Q) = a-bQ, Q=q_1+q_2$
		\item $C_i(q_i)= c_iq_i$
	\end{itemize}
\end{tcolorbox}

Empezaremos en $t=2$, se tiene que 

$$ \Pi_2 = (a-b(q_1+q_2))q_2-C_2q_2, $$

obteniendo la F.O.C llegamos a

$$ q_2 = \frac{a-C_2}{2b}-\frac{1q_1}{2} $$

Ahora en t=1

$$ \Pi_1 = (a-b(q_1+q_2))q_1-C_1q_1, $$

obteniendo la F.O.C llegamos a

$$ q_1=\frac{q+C_2-2C_1}{2b}$$

sustituyendo $q_1$ en $q_2$ se tiene que 

$$ q_2= \frac{a-3C_2+2C_1}{4b} $$

\section{Clase del 23 de septiembre}

\subsection*{Comparación de Stackelbeg y Cournot}

Veamos que 

$$\begin{matrix}
	\text{Stackelberg} & \text{Cournot}\\
	
	q_1= \dfrac{a-c}{2b} & q^c=\dfrac{a-c}{3b}\\
	
	q_2= \dfrac{a-c}{4b}
	
\end{matrix}
$$

\subsection*{Conclusiones:}

\begin{itemize}
	\item Competencia de Cournot: Firmas eligen $q$
	\item Competencia de Stackelberg: Firmas eligen $q$ secuencionalmente
	\item Competencia de Bertrand: Firmas eligen precios
\end{itemize}

\textbf{¿Qué modelo usar?} Si la capacidad y el nivel de producción se ajustan facilmente, Bertrand. Si la capacidad es dificil de ajustar, Cournot.

\begin{itemize}
	\item Excedente del consumidor: $CS^M<CS^C<CS^S<CS^{PC}$
	\item Excedente del productor: $PS^M>PS^C>PS^S>PS^{PC}$
\end{itemize}

\subsection*{Oligopolio Dinámico}

Bajo competencia oligopólica las acciones de las firmas generan externalidades competitivos.

\textbf{Colusión:} Ocurre cuando las firmas se ponen de acuerdo en un perfil de precios o cantidades que reducen las externalidades competitivas. La colusión también puede ocurrir en decisiones de inversión o publicidad. La colusión es ilegal en México, EE.UU, entre otros países.

\textbf{¿Cómo explicarias Bertrand a un estudiante de MBA?}

Guerra de precios: Si ambos tenemos $P>MC$ y yo bajo mis precios hoy, es de esperarse que mi rival baje su precio mañana y que el proceso siga hasta que $P=MC$.

\subsection*{Colusión en el oligopolio de Cournot}

Iniciamos con Cournot en un periodo, las firmas eligen cantidades simultáneamente y tienen costo constante unitario cero.

\textbf{Resultado no-cooperativo:}

\begin{itemize}
	\item Equilibrio de Nash en cantidades donde: 
		\begin{itemize}
			\item $q_1^C=\dfrac{a}{3b}$ \item $q_2^C=\dfrac{a}{3b}$ \item $Q=\dfrac{2a}{3b}$ \item $P=\dfrac{a}{3}$  
			\item $\Pi=\dfrac{a^2}{9b}$
		\end{itemize}
\end{itemize}

\textbf{¿Cúal sería el mejor resultado cooperativo?}

Si deciden cooperar maximizarán su función de beneficios conjunta.

$$ \Pi (q_1,q_2)= P(q_1,q_2)(q_1+q_2),$$

como las firmas son iguales $q_1+q_2=Q$ y cada firma produce: $\dfrac{Q}{2}$, entonces:

$$ \Pi (Q)= P(a-b)Q,$$

calculando $Q$, tal que $Q=\dfrac{a}{2b}$ se sustituye en el precio de tal forma que

$$ P=\dfrac{a}{2}$$

por lo que el beneficio es

$$ \Pi =  \dfrac{a^2}{8b}$$

el cuál es mayor que el equilibrio de Nash.

\subsection*{¿El equilibrio cooperativo es sostenible?}

No es sostenible en un juego estático, el único equilibrio de Nash es el resultado no cooperativo (Cournot)

\subsection*{¿Seguimos estando cuerdos?}

¿Qué ocurre si la firma $1$ elige el nivel de producción cooperativo? Si $q_1=\dfrac{a}{4b}$, ¿la firma 2 elige $q_2=\dfrac{a}{4b}$?

Las firmas buscan desviarse en un juego en una etapa, no hay espacio para la colusión, si una firma se desvía, no hay espacio para castigarla.

\section{Clase del 25 de septiembre}

\subsection*{Colusión en el ologopolio de Cournot: Parte 2}

Dado que el resultado cooperativo (colusivo) no es sostenible en un juego en una etapa, repetiremos el juego $T$ veces para permitir castigo por desviaciones. Suponga que las firmas juegan Cournot por $(t+1)$ periodos, $t>0$. En cada periodo $0\leq t \leq T$ las firmas eligen cantidades de forma no cooperativa. Las firmas eligen la estrategia de equilibrio que maximiza sus beneficios descontados:

\begin{itemize}
	\item Beneficios descontados:  $\sum_{t=0}^{T}\delta^t\Pi_t^i$
	\item Factor de descuento: $\delta$
	\item Beneficios de la firma i: $\Pi_t^i$
\end{itemize}

\subsection*{Concepto de equilibrio: Equilibrio perfecto en subjuegos}

\subsection*{Backward induction}

Empezando en $t=T$, se tiene que como el mundo termina al final del periodo $T$ las firmas juegan Cournot estático, no hay motivos para desviarse del equilibrio no cooperativo pues no habrá forma de castigar.

$$ q_1^c+q_2^c=\dfrac{a}{3b} $$

En $t=T-1$, las firmas saben que independientemente de sus acciones en $t=T-1$ el resultado no cooperativo se jugará en $T$, no hay amenaza de castigo y las firmas juegan Cournot no cooperativo.

$$ q_{T-1}^{1}=q_{T-1}^2=\dfrac{a}{3b} $$

Podemos hacer el mismo argumento para $T-2,T-3,\dots,0$. En cada etapa se jugará el equilibrio de Cournot estático. \textbf{La colusión no es sostenible en el juego de Cournot repetido un número finito de veces}.

\textbf{Rsultado final:} Tomando un juego en forma normal con un único equilibrio de Nash y repitiendolo $T+1$ veces. El único equilibrio perfecto en subjuegos del juego repetido es el equilibrio de Nash del juego en un periodo (repetido $T+1$ veces).

\subsection*{Colusión en el oligopolio de Cournout: Parte 3}

Dado que la colusión no es sostenible mi em Cournot estático ni en Cournot finito, probaremos qué ocurre en Cournot repetido infinitas veces.

Las firmas buscan maximizar son beneficios descontados:

$$\text{max}\sum_{t=0}^{\infty}\delta^i_t $$

El juego puede tener una variedad de equilibrios perfectos en subjuegos. Restringimos el análisis a estrategias simples que llamamos: "estrategias de gatillo". Sea

\begin{itemize}
	\item La producción de cada firma cuando no cooperan: $q_{NC}=\dfrac{a}{3b}$
	\item La producción de cada firma cuando se coluden (cooperan): $q_c=\dfrac{a}{4b}$
\end{itemize}

\textbf{Definimos las estrategias de gatillo:} La firma $i$ juega una estrategia de gatillo si $q_1^{0}=q_t$, firma i elige producción de colusión en $t=0$.

Para todo $t\geq1$: si $q_i^C=...$

\section{Clase del 26 de septiembre}

\subsection*{Colusión en el oligopolio de Cournot: Parte 3}

\begin{tcolorbox}[title=  Proposición]
	Existe $\delta^*$ al que el resultado en el que ambas firmas juegan sus estrategias de gatillo es un equilibrio perfecto en subjuegos si y solo si $\delta \geq \delta^*$.
\end{tcolorbox}

Como no se puede usar backward induction se tiene que calcular un resultado en equilibrio perfecto en subjuegos, este es si induce un equelibrio de Nash en cada subjuego del juego original.

Suponga que la firma $1$ juega su estrategia de gatillo, revisamos si en cada subjuego la mejor respuesta de la firma $2$ es jugar su estrategia de gatillo.

Hay dos tipos de subjuegos: 

\begin{enumerate}
	\item Los que inician en $t$ y alguna firma se desvía en $t'<t$.
	\item Los que inician en $t$ y no hubo desviaciones antes que $t$.
\end{enumerate}

\textbf{Consideremos el primer tipo de subjuegos:} Como la firma $1$ juega su estrategia de gatillo, elegirá $q_{NC}$ por siempre. Lo mejor que puede hacer la firma 2 es elegir $q_{NC}$, lo que es parte de su estrategia de gatillo de la firma 2.

El resultado en el que ambas firmas juegan sus estrategias de gatillo es equilibrio de Nash de subjuegos tipo $1$

\textbf{Consideremos el segundo tipo de subjuegos:} Como nadie se ha desviado en la historia y la firma $1$ juega su estrategia de gatillo, la firma 1 juega $q_C$ en $t=t$. \textit{¿Cuál es la mejor desviación posible para la firma 2?} 

Tengamos que Juega su mejor respuesta

$$ BR_2(q_1)=q_2=\dfrac{a-bq_1}{2b},$$

tomando en cuenta que $q_1=\dfrac{a}{4b}$ se tiene que 

$$ q_2=\dfrac{3a}{8b},$$

para el periodo t. En todos los periodos siguientes la \textit{firma 1} suelta el gatillo y juega $q_{NC}$. La mejor respuesta de la \textit{firma 2} es jugar también $q_{NC}$ (equilibrio de Cournot estático).

La firma 2 gana en los periodos $T\geq t+1$ 

$$ \Pi_{NC}= \dfrac{a^2}{9b},$$

lo máximo que gana la firma 2 por desviarse es:

$$ \Pi_D + \sum_{t=t+1}^{\infty}\delta^{T-t}\Pi_{NC} $$

si la firma 2 no se desvía gana

$$\sum_{T=t}^{\infty}\delta^{T-t}\Pi_C .$$

Si se tiene que

$$ \Pi_D + \sum_{t=t+1}^{\infty}\delta^{T-t}\Pi_{NC} \geq \sum_{T=t}^{\infty}\delta^{T-t}\Pi_C $$

Si la desigualdad se cumple, la firma 2 no se desvía en $t$ y sigue cooperando (juega estrategia de gatillo). En ese caso jugar estrategias de gatillo es equilibrio de Nash en los dos tipos tipos de subjuegos. \textit{¿Qué necesito para que esto ocurra?}

\section{Clase del 30 de Septiembre}

\subsection*{Colusión en el oligopolio de Cournot infinito}

En clases pasadas nos preguntábamos si es que ambas firmas jugaran sus estrategias de gatillo era uno de los posibles equilibrios perfectos en subjuegos de Cournot repetido infinitamente. Si este fuera el caso, la colusión sería sostenible.

Nos encontramos estudiando dos tipos de subjuegos:

\textbf{Tipo $1$:} Alguien había traicionado en la historia previa (encontramos que para ambas firmas era óptimo jugar sus estrategias de gatillo).

\textbf{Tipo $2$:} Nadie había traicionado en la historia previa (buscabamos el $\delta^*$ que haría óptimo jugar estrategias de gatillo comparando los beneficios descontados de traicionar y no traicionar).

La \textit{firma 2} no quiere desviarse syss:

$$ \Pi_D + \sum_{\tau=t+1}^{\infty}\delta^{\tau-t}\Pi_{NC}\leq\sum_{\tau=t}^{\infty}\delta^{\tau-t}\Pi_C $$

en el lado izquierdo siendo los beneficios de traicionar en t (no jugar estrategias de gatillo en subjuegos tipo 2) y a la derecha los beneficios de no traicionar o desviarse (Jugar estrategias de gatillo en subjuegos tipo 2).

El factor de descuento que hace sostenible la colusión (hace que ambas firmas jueguen sus estrategias de gatillo en ambos tipos de juegos) es:

$$ \delta\geq \dfrac{9}{17} = \delta^* $$

esto siendo el factor crítico de descuento. Si las firmas son suficientemente pacientes, la colusión es factible en el oligopolio de Cournot repetido infinitas veces. De forma intuitiva, cuando las firmas deciden si desviarse del acuerdo colusivo enfrentan el siguiente trade-off:

\begin{itemize}
	\item Desviarse hoy permite capturar más beneficios a corto plazo.
	\item Pero esto genera un castigo infinito mañana, si las firmas le dan alta importancia al futuro (alta $\delta$), el segundo efecto domina y la colusión se obliga a ocurrir.
	\item Un parámetro facilita la corrupción disminuye $\delta^*$
\end{itemize}

\subsection*{Colusión en el oligopolio de Bertrand infinito}

Supongamos ahora que las firmas compiten en precios sea $Q(P)$ la demanda total. Considere el juego en una etapa, ¿cuál es el mejor resultado cooperativo?

Jugar $P=P^M$, donde $P^M$ resuelve:

$$ \text{arg. max} (P-C)Q(P),$$

cada firma gana $\frac{\Pi^M}{2}$, con:

$$ \Pi^M = (P^M-C)Q(P^M).$$

Si una firma se desvía del resultado cooperativo, ¿cuántos beneficios gana?

\begin{itemize}
	\item Mejor desviación:
	\begin{itemize}
		\item $P=P^M-\epsilon$
		\item $\Pi_D= \Pi^M$
	\end{itemize}
	\item ¿Cuál es el resultado no cooperativo?
	\begin{itemize}
		\item $P_M=MC=C$
		\item $\Pi_{NC}=0$
	\end{itemize}
\end{itemize}


\section{Clase del 2 de octubre}

\subsection*{Colusión en el ologopolio de Bertrand}

Suponga que la \textit{firma $2$} está jugando su estrategia de gatillo y que la \textit{firma $1$} elige el precio $p^M$ en todos los periodos anteriores a $t$. La mejor desviación posible para la \textit{firma $1$} es $P^M-\epsilon$ y ganar $\Pi^D$ en $t$, en los siguientes periodos jugara $P_1=c$. Para que la colusión sea sostenible como equilibrio, esta desviación debe ser menos rentable que mantenerse cooperando.

Siendo los beneficios de la forma

$$ \Pi_D + \sum^{\infty}_{\tau =t+1}\Pi_{NC} \leq \sum^{\infty}_{\tau=t}\delta^{\tau-t}\Pi_C,$$

para un $\delta\geq \frac{1}{2}$, la colusión (equilibrio en estrategias de gatillo) es equilibrio perfecto en subjuegos si y solo si $\delta\geq \frac{1}{2}$, las firmas se coluden si son suficientemente pacientes.

\subsection*{Determinantes de Colusión}

\subsubsection*{Número de Firmas}

Suponga que hay $N$ firmas compitiendo á la Bertrand, el beneficio de cada firma por coludirse es $\frac{\Pi^M}{N}$, el beneficio de desviarse es $\Pi^M$ (después del periodo en que me desvío es cero). Teniendo en cuenta que 

$$ \Pi_D + \sum^{\infty}_{\tau =t+1}\Pi_{NC} \leq \sum^{\infty}_{\tau=t}\delta^{\tau-t}\Pi_C,$$

se sigue que

$$ \Pi^M \leq \sum^{\infty}_{\tau=t}\delta^{\tau-t}\frac{\Pi^M}{N},$$

para un $\delta\geq\frac{N-1}{N}$, el $\delta^*$ que hace sostenible la colusión es creciente en el número de firmas. 

El \textuparrow $N$ reduce los beneficios de coludirse, aumentando los incentivos a desviarse.

Alcanzar un acuerdo colusivo es más complicado si hay más participantes.

La probabilidad de que la autoridad detecte una asociación colusiva aumenta con más participantes.

\subsubsection*{Crecimiento de mercado}

Regresamos al caso de Bertrand con dos firmas: supongamos ahora que la demanda al tiempo $t$ es: $Q_t(P)=A^tQ(P)$ con $A>0$. 

Los beneficios por desviarse son: $\Pi_D(t)=A^t\Pi^M$, los beneficios no cooperativos son $\Pi_{NC}=0$, la condición de no desviación es: 

$$ \Pi^M + \sum_{\tau=t+1}^{\infty}\delta^{\tau-t}\Pi_{NC}\leq \sum_{\tau=t}^{\infty}\delta^{\tau-t}A^{\tau-t}\frac{\Pi^M}{2} $$

considerando un $\delta \geq \frac{1}{2A}$. La colusión es más facil en un mercado en expansión que en uno que se contrae.


\section{Clase del 3 de octubre}

\subsection*{Número de cambios de precio al año}

Ahora suponga que las firmas cambian sus precios $f$ veces al año, el periodo de tiempo es $\frac{1}{f}$ del año.

Suponemos que $\delta = \frac{1}{1+r}$, en este caso:

$$\delta = \frac{1}{1+\frac{r^y}{f}}$$

En el modelo de Bertrand con dos firmas la colusión solo es factible si $\delta\geq\frac{1}{2},$ esto ocurre syss:

$$\dfrac{1}{1+\frac{r^y}{f}}\geq \dfrac{1}{2}$$

Despejando $f$ de la desigualdad se tiene que $f\geq r^y$, se concluye que la colusión es más fácil cuando se ajustan los precios con frecuencia. Cuando las firmas interactúan con frecuencia pueden detectar las desviaciones del acuerdo colusivo rápidamente.

\subsection*{Recortes inobservables de Precios}

Cuando las firmas están imperfectamente informadas de los precios rivales, la colusión es más complicada, una firma que se desvía podría no ser castigada si los rivales no observaron la disociación.

\subsection*{Soluciones para facilitar la colusión}

\textbf{Igualamos el precio de la competencia:} Los consumidores pueden pedir reembolsos si el rival tiene mejores precios (Consumidores monitorizan el cumplimiento del acuerdo colusivo)

\textbf{Consumidor más beneficiado:} Consumidores Pasados pueden pedir reembolsos si bajas el precio (Reduce el incentivo a bajar precios)

\begin{tcolorbox}[title= Ejemplo]
	En alguna ocasión el gobierno de Dinamarca decidió publicar precios de transacción de compra de concreto entre agentes, facilitando la colusión, el precio subió un $20\%$ ese año y dejó de publicar precios.
\end{tcolorbox}

\subsection*{Colusión con firmas asimétricas}

Suponga que las \textit{firmas 1 y 2} tienen gastos asimétricos $c_1=c<\bar{c}= c_2$ puede mostrarse que si $\delta \leq \frac{1}{2}$ existen equilibrios colusivos con características:

$$ P \in [P^M(c),P^M(\bar{c})] $$

con market shares $S_i>0$ para $i=1,2$.

\textbf{Problemas:} No hay una forma simple de elegir entre equilibrios, es más dificil elegir entre el precio colusivo y la segmentación de mercado, hay incentivos para que las firmas mientan en sus costos marginales, las firmas no satisfechas pueden iniciar guerras de precios. El resultado de maximización de beneficios conjunta no es equilibrio perfecto en subjuegos. 

\textbf{Soluciones:} Sobornos

\begin{itemize}
	\item Firma 1 oferta a todo el mercado a $ P^M(c) $
	\item Firma 2 elige $P \rightarrow \infty$
\end{itemize}

Siempre que la firma 2 no esté en el mercado, la firma 1 la soborna. Problema: ¿a cuánto debe ascender el soborno? ¿Cómo ocultarlo?

Por lo que la colusión es más difícil con firmas asimétricas.

\chapter{Ejercicios}

\section{Primera lista de ejercicios}

\subsection*{Ejercicio I}

\end{document}