
%Tipo de documento
\documentclass[letterpaper,12pt,twocolumn]{report}

%Margenes
\usepackage[top=2.5cm, bottom=2.5cm, left=2.5cm, right=2.5cm]{geometry}

%Paquetes variados
\usepackage[T1]{fontenc}
\usepackage[utf8]{inputenc}
\usepackage[spanish]{babel}
\usepackage{lmodern}
\usepackage{amsmath}
\usepackage{amsfonts}
\usepackage{amssymb}
\usepackage{amsthm}
\usepackage{graphicx}
\usepackage{color}
\usepackage{xcolor}
\usepackage{url}
\usepackage{theorem}
\usepackage{textcomp}
\usepackage{hyperref}
\usepackage{parskip}
\usepackage{fancyvrb}
\usepackage{tikz}
\usepackage[most]{tcolorbox}
\usepackage{tabularray}
\usepackage{mathtools} 

%Diagramas de arbol
\usepackage{forest}

%Capitulos elegantes
\usepackage[Rejne]{fncychap}

%Datos 
\title{Notas de Clase: Organización industrial}
\author{Fabián Méndez Martinez}
\date{Agosto 2024 - Enero 2025}
%Comandos
\renewcommand{\baselinestretch}{1.3}

%Documento
\begin{document}

%Presentación
\maketitle
\tableofcontents

%Contenido

\chapter{Introducción}


%Datos de la materia 
\begin{tcolorbox}[title=\large Información de la EE]
    \begin{itemize}
        \item Nombre: Organización industrial
        \item NRC: 92681
        \item Clave: ECTE - 38008
        \item Créditos: 8
    \end{itemize}
\end{tcolorbox}

%Profesor
\begin{tcolorbox}[title=\large Información del Profesor]
    \begin{itemize}
        \item Profesor: Miguel Rodrigo Ávila
Pulido
        \item Correo: miavila@uv.mx
    \end{itemize}
\end{tcolorbox}

%horario
\begin{table}[h]
\centering
\begin{tabular}{l|ll}
\multicolumn{3}{c}{\textbf{Horario}}      \\ 
\hline
Martes     & 7:00 & 9:00          \\ 
\hline
Jueves & 7:00 & 9:00          \\ 
\hline
Viernes   & 9:00 & 11:00          \\ 
\hline
\multicolumn{3}{c}{Aula: 217}   
\end{tabular}
\end{table}
\pagebreak

%Clases
\section{Clase del 19 de agosto}

\subsection*{Organización industrial}

La organización industrial estudia la competencia imperfecta. En las economías de mercado se tiene que 

\begin{itemize}
    \item Las firmas deciden qué y cuanto producir
    \item Los consumidores eligen comprar en la mejor alternativa por precio, calidad, ubicación, etc.
\end{itemize}

Este curso estudia el comportamiento de las empresas.

\subsection*{Historia de la Organización industrial}

\subsubsection*{Harvard}

\textbf{Caso: Demanda del gobierno de EE.UU contra \textit{US Steel}:} La firma concentraba el $70\%$ del mercado de producción de acero. El gobierno americano perdió la demanda, \textit{US Steel} no violó la ley de competencia pues \textit{"La Ley no hace al tamaño una ofensa"}.

¿Comó inferis comportamientos ilegales a partir de características estructurales como el tamaño? A partir de esto se creo el siguiente marco

\begin{tcolorbox}[title=Definición]
$\text{Marco}=\text{Estructura}\rightarrow \text{Conducta}\rightarrow \text{desempeño}$
\end{tcolorbox}


La estructura determina la forma en la que las firmas interactuan entre sí, con los compradores y los potenciales entrantes.

\begin{tcolorbox}[title=Definición]
Estructura $ = f$(Firmas, tecnología, productos, etc.)
\end{tcolorbox}

La estructura determina la conducta (forma en la que las firmas se comportan en una estructura de mercado dada). La conducta determina el desempeño (Beneficios, excedentes, bienestar)

Señalan que la alta concentración de mercado reduce el excedente de los consumidores.

\textbf{Debilidades:} Suponer que la concentración es exógena, no tomar en cuenta diferencias interindustriales.

\subsubsection*{Chicago}

Está de acuerdo con \textbf{Harvard} en que las firmas con mayor concentración de mercado tienden a generar mayores beneficios. Para \textbf{Harvard} esto implica que los mercados concentrados son menos competitivos.

Para \textbf{Chicago} este no es el caso, puede ser que las firmas con mayor de mercado sean más eficientes y por ello tengan mayores beneficios.

\textbf{Énfasis en teoría de precios:} Afirman que el comportamiento monopólico es dificil de confirmar, no ocurre con frecuencia y cuando ocurre es tipicamente transitorio.

\subsubsection*{Post Chicago}

Énfasis en la toma de decisiones estratégicas, el comportamiento de las firmas se representa con modelos matemáticos y de toería de Juegos. Se construye un marco formal que supera el debate entre \textbf{Chicago} y \textbf{Harvard}.

También surge nueva literatura empírica que combina los modelos matemáticos con pruebas econométricas (\textit{Ordered-probit, multinomial-logit, random coefficient logit, etc.}) computacionalmente demandantes.

\textbf{Dificultades:} ¿Qué modelos usar para cada caso? Efectos de segundo orden.

\subsection*{Competencia Perfecta}

\begin{tcolorbox}[title= Supuestos]

Un agente se comporta de forma competitiva si:

\begin{itemize}
    \item Suponen que el precio de mercado está dado.
    \item Bienes homogeneos.
    \item Libre entrada y salida, por ejemplo los costos fijos pueden ser considerados una barrera de entrada.
    \item Información perfecta.
    \item No hay externalidades.
    \item Bienes perfectamente divisibles.
    \item No hay costos de transacción.
\end{itemize}

\end{tcolorbox}


\section{Clase del 21 de agosto}

\subsection*{El Benchmark de competencia perfecta}

\textbf{Objetivo de la firma:} Maximizar beneficios por lo que se tiene que 

\begin{tcolorbox}[title=Precio en competencia perfecta]
    $$\text{max}=pq-C(q)$$
Obteniendo F.O.C
    $$ p-C'(q)=0 $$
    $$ p=C'(q)$$
    $$ p=MC(q) $$
\end{tcolorbox}

Se tienen los siguientes funciones de costo

\begin{itemize}
    \item Costo total: $C(q)=c(q)+F$
    \item Costo variable: $c(q)$
    \item Costo medio: $AC=\frac{c(q)}{q}+\frac{F}{q}$
    \item Costo marginal: $MC= C'(q)$
    \item Costo variable medio: $AVC=\frac{c(q)}{q}$
    \item Costo fijo medio: $AFC=\frac{F}{q}$
\end{itemize}



% Aquí va la gráfica 1.....


\textbf{Demuestre} que la curva de costo marginal cruza la curva de costo medio en su mínimo.

Sea $\hat{Q}$ la cantidad que minimiza $AC:\frac{dAC(\hat{Q})}{d\hat{Q}}=0$ se tiene que 

$$
\dfrac{AC(Q)}{dQ}= \frac{dC(Q)}{dQ}\cdot Q^{-1}+ C(Q)(-1)Q^{-2},
$$

$$ 
\vdots
$$

$$
\dfrac{AC(Q)}{dQ} = \dfrac{\frac{dC(Q)}{dQ}Q-C(Q)}{Q^{2}}.
$$

Si se tiene que $Q=\hat{Q}$, se cumple que 

$$ \frac{dAC(\hat{Q})}{d\hat{Q}}=C(\hat{Q}),$$

por lo que 

$$ \dfrac{AC(Q)}{dQ} = \dfrac{C(\hat{Q})-C(\hat{Q})}{\hat{Q}^{2}} = 0.$$

\subsection*{Mercado competitivo}

\subsubsection*{Equilibrio de corto plazo}

Pensemos en $n$ firmas idénticas y que los costos fijos son costos fijos son costos hundidos en el corto plazo. La oferta de mercado de corto plazo es la suma horizontal sw curvas de oferta de cada firma

% Aquí van las gráficas 2 y 3 side by side

En el corto plazo puede ser el caso que una firma representativa tenga beneficios mayores a cero.

\subsubsection*{Equilibrio de largo plazo}

La entrada de firmas al mercado en el largo plazo se detiene cuando el precio llega al mínimo del costo promedio se largo plazo.

% Aquí va las gráficas 4 y 5

\section{Clase del 22 de agosto}


\subsection*{Recordatorio de bienestar y excedentes}

Se define al \textbf{Bienestar} como la suma de los excedentes de todos los consumidores y de todos los productores. El \textbf{excedente del consumidor} es el área bajo la curva inversa de demanda y por arriba del precio (i.e diferencia entre lo que un consumidor esta dispuesto a pagar y el precio que efectivamente pagó multiplicado por la cantidad consumida).


El \textbf{Excedente del productor} es la área de la curva de oferta y abajo del precio.


% Aquí va la gráfica 5

\subsection*{Teoremas del bienestar}

\begin{tcolorbox}[title= Primer teorema del bienestar]
   
   Es cualquier asignación descentralizada obtenida a tráves de un mercado libre es eficiente en el sentido de pareto y maximiza el bienestar social.
   
\end{tcolorbox}

\begin{tcolorbox}[title= Segundo teorema del bienestar]
   
   Es cualquier asignación en la curva de contratos es alcanzable a través de la redistribución de las asignaciones iniciales.
   
\end{tcolorbox}

\subsection*{Teoría de Juegos }

Estudia los problemas de decisión multi-agente. En competencia imperfecta hay interacciones estratégicas entre agentes. 

\subsubsection*{Juego en forma normal}

\begin{enumerate}
    \item Conjunto de Jugadores $I=\{1,2,\dots,N\}$
    \item Para cada jugador un conjunto de estrategias $A_i$.
\end{enumerate}

De esta ultima definimos que sea $a=(a_1,a_2,\dots,a_n)$ una lista de las acciones de cada jugador. Decimos que $a$ es un perfil de acciones.

\begin{enumerate}
	\setcounter{enumi}{2}
    \item Para cada jugador una función de pago.
\end{enumerate}


\section{Clase del 26 de Agosto}

\subsection*{Ejemplo de juego en forma normal}

Suponga que en competencia en precios cuando los bienes son sustitutos perfectos

\begin{enumerate}
	\item Conjunto de Jugadores: $I: \{f1, f2\}$
	\item Conjunto de estrategias: $ P_i \in R_{++}$
	\item Conjunto de funciones de pagos: $\Pi_i(P_i,P_j) = (P_i-C_i)q_i(P_i,P_j)$
\end{enumerate}

\subsection*{Dilema del prisionero}

\begin{enumerate}
	\item Conjunto de Jugadores: $I: \{\text{Prisionero }1,\text{Prisionero } 2\}$
	\item Conjunto de estrategias: $ S_i= \{ \text{Delatar, no delatar} \}$ para $i \in I$
	\item Conjunto de funciones de pagos: 
	
	$ U_i(S_i,S_2) \begin{cases}
		-5 & \text{ si } S_i=M,S_j=F \\
		 -4 & \text{ si } S_i=S_j=F\\
		  -2 & \text{ si } S_i=S_j=M\\
		   -1 & \text{ si } S_i=F,S_j=M
	\end{cases}
	$
\end{enumerate}

\subsection*{Representación matricial de Juegos Finitos con dos jugadores}

\textbf{Convenciones:} Las filas representan al jugador 1, las columnas al jugador 2 y en las entradas los pagos.

\subsubsection*{Matriz del dilema del prisionero}
$$
\begin{matrix}
	 &  F & M \\
	F & (-4,-4) & (-1,-5)  \\
	M & (-5, -1) & (-2,-2)
	 
\end{matrix}
$$

\subsection*{Equilibrio en estrategias dominantes}

Sea $s\in \Pi^{N}_{j=1}\cdot S_j$ para $i\in\{1,\dots, N \}$ denotemos a $S_{-i} = (S_i,S_2,\dots,S_{i-1},S_{i+1},\dots ,S_n)$ y reescribimos el conjunto de acciones $S$ como $S= (S_i,S_{-i} )$.

En los juegos en forma normal una estrategia es una acción que pertenece al conjunto de acciones de algún jugador.

\begin{tcolorbox}[title= Definición]
	Si para todo $S_{-i} \in \Pi_{j\neq 1}S_J$ y para todo $s_i \in S_i$ sujeto a que $ s_i \neq \widetilde{s_i} $
	
	$$ \Pi_i(\widetilde{s_i},s_{-i})> \Pi_i(s_i,s_{-1}) $$

Se tiene que $\widetilde{s_i} \in S_i$ es una estrategia estrictamente dominante para el jugador $i$.
	
\end{tcolorbox}

 
En el dilema del prisionero la estrategia estrictamente dominante de ambos jugadores es delatar (F).

\subsection*{Una primera definición de equilibrio}

\begin{tcolorbox}[title= Primera definición de equilibrio]
	Un perfil de estrategias $s=(\widetilde{s_1},\widetilde{s_2}, \dots \widetilde{s_n}) \in \Pi_{j=1}^N S_j$ es un equilibrio en estrategias dominantes si $\widetilde{S_i}$ es una estrategia dominante para el jugador $i$.
\end{tcolorbox}

Para el dilema del prisionero (F,F), $U_i=-4$ para $i\in I$. Por lo que no hay razón para que las desiciones de mercado lleven a un óptimo de pareto.

\subsubsection*{La batalla de los sexos}


$$
\begin{matrix}
	&  Opera & Football \\
	Opera & (2,1) & (0,0)  \\
	Football & (0, 0) & (1,2)
	
\end{matrix}
$$

No hay equilibrio en estrategias estrictamente dominantes.

\begin{tcolorbox}[title= Segunda definición de equilibrio]
	\textbf{Equilibrio de Nash}: Un perfil de estrategias $\widetilde{S}=(\widetilde{S_1},\widetilde{S_2}, \dots, \widetilde{S_n}) \in \Pi_{j=1}^N S_j$ es un equilibrio de Nash si ningún jugador tiene incentivos a desviarse de este perfil estratégico. Esto es: si para todo $i \in I$ para todo $s_i \in S_i$, 
	
	$$ \Pi(\widetilde{S_i},\widetilde{S_{-i}})\geq \Pi_i(S_i,\widetilde{S_{-i}}) $$
\end{tcolorbox}

\section{Clase del 28 de agosto}

\subsection*{¿Cómo encontrar el equilibrio de Nash?}

Tengamos un juego simple $2x2$ con una matriz 

$$
\begin{matrix}
	&  C & D \\
	C & (-1,-1) & (-10,0)  \\
	D & (0, -10) & (-6,-6)
	
\end{matrix}
$$

Por descripción tengamos que 

\begin{itemize}
	\item $(C,C)$: No es un equilibrio de Nash, ambos jugadores no tienen incentivos para jugar \textit{D}.
	
	\item $(C,D)$: No es equilibrio de Nash, el jugador $1$ tiene incentivos para jugar \textit{D}.
	
	\item $(D,C)$: No es equilibrio de Nash, el jugador $2$ tiene incentivos para jugar \textit{D}.
	
	\item $(D,D)$: Sí es equilibrio de Nash.
\end{itemize}


\subsubsection*{Ejemplo}

$$
\begin{matrix}
	&  C & D \\
	C & (2,1) & (0,0)  \\
	D & (0,0) & (1,2) 
	
\end{matrix}
$$

\begin{itemize}
	\item $(O,O)$: Sí es equilibrio de Nash

	\item $(F,F)$: Sí es equilibrio de Nash
\end{itemize}


Si el juego más complejo, resolverlo por inspección tomaría mucho tiempo.

\subsubsection*{Análisis de Mejor Respuesta}

\begin{itemize}
	\item Sea $i \in \{1,2,\dots , N \}$ y $S_{-i} \in \Pi_{j\neq1} S_j$
	\item Sea $BR_i(S_{-i})=\text{ arg.max } U_i(s_i,S_{-i})$
	\item Si $S_{-i} \in \Pi_{j\neq1} S_j \rightarrow \text{ Si }BR_i(S_i)$ es la correspondencia de mejor respuesta del jugador $i$.
\end{itemize}

\begin{tcolorbox}[title= Nota]
	
Varios valores de $s_i$ podrían resolver el problema de maximización. \textit{arg.max} es el conjunto de valores de "$x$" para los que la función alcanza su valor máximo.

Un perfil estratégico $$\widetilde{S}=(\widetilde{s_1},\widetilde{s_2},\dots, \widetilde{s_N})\in \Pi_{j=1}^N S_j,$$ es un equilibrio de Nash si para todo i: 

$$ \widetilde{S_i}\in BR_i(S_{-i}). $$

\end{tcolorbox}


Cada jugador elige una estrategia que es su mejor respuesta a las estrategias de equilibrio de sus rivales.

Para resolver un juego encontrando el equilibrio de Nash con análisis de mejor respuesta:

\begin{enumerate}
	\item Para cada jugador $i$, buscar la correspondencia de mejor respuesta $BR_i(\cdot)$
	\item Buscar las intersecciones de las mejores respuestas, con esto obtendrás todos los equilibrios de Nash.
\end{enumerate}

\subsubsection*{Ejemplo}
Considere la competencia en precios con productos diferenciados $c_1=c_2=0$.
Sea la función de pagos $\Pi_i(P_i,P_j)=(P_i-C_i)q_i(P_i,P_j),$ suponga $q_i=1-2P_i+P_j$. Encuentre el equilibrio de Nash:

%No sé si el uso de align sea la mejor solución

Partiendo desde la función de pagos
\begin{align*}
	\Pi_i(P_i,P_j) &=(P_i-C_i)\cdot q_i(P_i,P_j),\\
	\shortintertext{sustituyendo $q_i$}
	\Pi_i(P_i,P_j) &=(P_i-C_i) \left( 1-2P_i+P_j\right),\\
		\shortintertext{considerando $C_i=0$}
	\Pi_i(P_i,P_j) &= P_i \left( 1-2P_i+P_j\right) ,\\
	\Pi_i(P_i,P_j) &= P_i-2P_i^2+P_iP_j.
\end{align*}
\vspace{.75cm}
Calculando las F.O.C
\begin{equation*}
	\dfrac{\partial \Pi_i(P_i,P_j)}{\partial q_i} = 
	1-4P_iP_j=0,
\end{equation*}

se sigue que

\begin{align*}
1-4P_iP_j&=0, \\
4P_i+P_j&=1, \\
4P_i&=1-P_j,\\
P_i&=\frac{1-P_j}{4}.
\end{align*}
Por lo que la función de mejor respuesta esta dada por 

$$ BR_i= \frac{1-P_j}{4}.$$


El proceso es análogo para el caso de $P_j$, sustituyendo $P_j$ en $P_i$

\begin{align*}
	P_i&=\frac{1-\left( \frac{1-P_i}{4} \right) }{4},\\
	P_i&=\frac{\frac{4-1+P_i}{4} }{4},\\
	P_i&=\frac{3+P_i}{16},
\end{align*}

% Separado para que haya una mejor distribución

\begin{align*}
	P_i&=\frac{3+P_i}{16},\\
	16P_i&=3+P_i,\\
	15P_i&=3,\\
	P_i &= \dfrac{3}{15}=\dfrac{1}{5}.
\end{align*}

Análogamente se se observa que 

$$ P_i = P_j = \dfrac{1}{5}. $$

\section{Clase del 29 de agosto}

\subsection*{Comentarios}

\begin{enumerate}
	\item Un equilibrio en estrategias dominantes es siempre un equilibrio de Nash.
	\begin{itemize}
		\item Una estrategia dominante es la mejor respuesta a cualquier perfil estratégico.
		\item Un equilibrio de Nash no siempre es equilibrio en estrategias dominantes.
	\end{itemize}
	\item Al buscar el equilibrio de Nash, solo considera desviaciones unilaterales del perfil estratégico.
\end{enumerate}


\begin{tcolorbox}[title=Fundamentos del equilibrio de Nash]
	\begin{itemize}
		\item Todos los agentes son racionales y todos los agentes saben que son racionales.
		\item Como los agentes son racionales, harán lo posible por maximizar su utilidad dadas las acciones de sus rivales.
		\item Al final se jugará un perfil estratégico del por que ningún agente se desviará.
	\end{itemize}
\end{tcolorbox}

\subsubsection*{¿Cómo elegir entre varios equilibrios de Nash?}

\begin{itemize}
	\item Ocasionalmente uno es Pareto dominante
	\item Otras veces uno actúa como punto focal
\end{itemize}

En juegos repetidos, el proceso de aprendizaje lleva a un equilibrio de Nash, incluso sin agentes racionales. La historia, dependiendo del punto de partida, podrías converger a equilibrios de Nash distintos.
$$
\begin{matrix}
  &	Piedra & Papel & Tijera \\
  Piedra & (0,0) & (-1,1) & (1,-1) \\
  Papel & (1,-1) & (0,0) & (-1,1) \\
  Tijera & (-1,1) & (1,-1) & (0,0)
\end{matrix}
$$

\textbf{No hay equilibrio de Nash en estrategias puras}
$$
\begin{matrix}
	&	Divide & Roba \\
	Divide & (\frac{1}{2},\frac{1}{2}) & (0,1) \\
	Roba & (1,0) & (0,0)
\end{matrix}
$$

En \textit{(Divide, Roba), (Roba, Divide), (Roba, Roba)}.

\subsection*{Juegos de forma extensiva}
\subsubsection*{Juego del piloto vs terrorista}

% Poner esa coma fue un martirio

\begin{forest}
	[El piloto mueve [Volar a cuba [Bomba [$(1\text{,}-1)$] ] [No bomba[$(1\text{,}1)$]]][Volar a NYC[Bomba[($-1\text{,}-1)$] ] [No Bomba [$(2\text{,}0)$]]]]
\end{forest}


$$
\begin{matrix}
	& BnNB & NB,B & B,B & NB,NB \\
	Cuba & (-1,-1) & (1,1) & (-1,-1) & (1,1)\\
	NYL & (2,0) & (-1,-1) & (-1,-1) & (2,0)
\end{matrix}
$$

\subsection*{Equilibrios de Nash}

\begin{itemize}
	
	\item \textit{(Cuba,(NB,B))}
	\item \textit{(NYL,(B,NB))}
	\item \textit{NYL(NB,NB)}
\end{itemize}


\section{Clase del 2 de septiembre}

Consideremos el primer equilibrio, un piloto racional diría, si vuelo a NYC, es interés del terrorista cambiar a \textit{NB} en lugar de \textit{B}. Este razonamiento no es capturado por el equilibrio de Nash:

\begin{center}
\noindent \textbf{Algunos equilibrios de Nash se sostienen por amenazas no creíbles.}
\end{center}

Necesitamos un concepto de equilibrio que elimine amenazas no creíbles.

\subsection*{Inducción hacia atrás (Backward induction)}

\begin{enumerate}
	\item Empieza resolviendo para las desiciones óptimas en el nodo terminal, encuentre los pagos.
	\item Vaya un paso atrás, resuelva para las desiciones óptimas, anticipando que todos se comportan racionalmente en nodos subsecuentes, encuentre los pagos.
	\item Itere hasta alcanzar el modo inicial.
\end{enumerate}

% Aquí va el la solución del arbol


\textbf{Equilibrio:} \textit{(NYC, (NB,NB))}.

\subsection*{Equilibrio Perfecto en subjuegos}

\begin{tcolorbox}[title=Definición]
	Un subjuego es un nodo de decisión del juego original con los nodos de decisión y los nodos terminales que siguen directamente a este nodo.
\end{tcolorbox}

Un subjuego es llamado subjuego estricto si es distinto del juego original.

\begin{tcolorbox}[title=Definición]

Un perfil estratégico es un equilibrio perfecto en subjuegos si induce un equilibrio Nash en cada subjuego del juego original.

\end{tcolorbox}

\subsubsection*{Suponga un juego finito:}
Incluso si los jugadores mueven simultáneamente el juego puede resolverse por \textit{backward induction}.

\begin{enumerate}
	\item Inicie por los subjuegos más profundos y encuentre el equilibrio de Nash en ellos.
	\item En la forma extensiva, remplaza los subjuegos más pequeños por los pagos del equilibrio de Nash.
	\item Itera hasta que no queden subjuegos.
\end{enumerate}

 \subsection*{Resumen}
 
 \begin{tcolorbox}[title= Juego en forma normal]
 	Tres conjuntos \textit{(jugadores, acciones, pagos)}
 	
 	\begin{itemize}
 		\item Solución: Equilibrio perfecto en subjuegos / equilibrio de Nash
 		\item Prueba y error / función de mejor respuesta
 	\end{itemize}
 	
 \end{tcolorbox}
 
 \begin{tcolorbox}[title= Juegos en forma extensiva]
 	\begin{itemize}
 		\item Concepto de solución: equilibrio perfecto en subjuegos (elimina amenazas no creíbles)
 		\item Método: Con periodos finitos usar inducción hacia atrás
 	\end{itemize}
 \end{tcolorbox}

\chapter{.-.}

\end{document}